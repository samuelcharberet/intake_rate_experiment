% Options for packages loaded elsewhere
\PassOptionsToPackage{unicode}{hyperref}
\PassOptionsToPackage{hyphens}{url}
%
\documentclass[
  12pt,
]{article}
\usepackage{amsmath,amssymb}
\usepackage{iftex}
\ifPDFTeX
  \usepackage[T1]{fontenc}
  \usepackage[utf8]{inputenc}
  \usepackage{textcomp} % provide euro and other symbols
\else % if luatex or xetex
  \usepackage{unicode-math} % this also loads fontspec
  \defaultfontfeatures{Scale=MatchLowercase}
  \defaultfontfeatures[\rmfamily]{Ligatures=TeX,Scale=1}
\fi
\usepackage{lmodern}
\ifPDFTeX\else
  % xetex/luatex font selection
\fi
% Use upquote if available, for straight quotes in verbatim environments
\IfFileExists{upquote.sty}{\usepackage{upquote}}{}
\IfFileExists{microtype.sty}{% use microtype if available
  \usepackage[]{microtype}
  \UseMicrotypeSet[protrusion]{basicmath} % disable protrusion for tt fonts
}{}
\makeatletter
\@ifundefined{KOMAClassName}{% if non-KOMA class
  \IfFileExists{parskip.sty}{%
    \usepackage{parskip}
  }{% else
    \setlength{\parindent}{0pt}
    \setlength{\parskip}{6pt plus 2pt minus 1pt}}
}{% if KOMA class
  \KOMAoptions{parskip=half}}
\makeatother
\usepackage{xcolor}
\usepackage[margin=1in]{geometry}
\usepackage{graphicx}
\makeatletter
\def\maxwidth{\ifdim\Gin@nat@width>\linewidth\linewidth\else\Gin@nat@width\fi}
\def\maxheight{\ifdim\Gin@nat@height>\textheight\textheight\else\Gin@nat@height\fi}
\makeatother
% Scale images if necessary, so that they will not overflow the page
% margins by default, and it is still possible to overwrite the defaults
% using explicit options in \includegraphics[width, height, ...]{}
\setkeys{Gin}{width=\maxwidth,height=\maxheight,keepaspectratio}
% Set default figure placement to htbp
\makeatletter
\def\fps@figure{htbp}
\makeatother
\setlength{\emergencystretch}{3em} % prevent overfull lines
\providecommand{\tightlist}{%
  \setlength{\itemsep}{0pt}\setlength{\parskip}{0pt}}
\setcounter{secnumdepth}{-\maxdimen} % remove section numbering
\ifLuaTeX
  \usepackage{selnolig}  % disable illegal ligatures
\fi
\usepackage{bookmark}
\IfFileExists{xurl.sty}{\usepackage{xurl}}{} % add URL line breaks if available
\urlstyle{same}
\hypersetup{
  pdftitle={Ingestion rate affects mass and nutrient balance and allocation},
  pdfauthor={Samuel Charberet},
  hidelinks,
  pdfcreator={LaTeX via pandoc}}

\title{Ingestion rate affects mass and nutrient balance and allocation}
\author{Samuel Charberet\footnote{Sorbonne Université, CNRS, UPEC, CNRS,
  IRD, INRA, Institut d'écologie et des sciences de l'environnement,
  IEES, F-75005 Paris, France.}}
\date{23 mai 2024}

\begin{document}
\maketitle

\section{Introduction}\label{introduction}

Elements cycles in ecosystems are impacted by animal communities in
various ways. One is the consumption of a resource, its use for growth
and maintenance, and the release of wastes, a process which might
accelerate or slow down the dynamics depending on the demographic and
physiologic characteristics of communities (De Mazancourt et al.~1998).
The quantity and quality of the consumed resource affect growth,
maintenance, and the release of wastes in both quality and quantity in a
process that depends on individuals and species level traits
(e.g.~digestive physiology). Although the effect of resource quality on
organismal and wastes chemical composition has been thoroughly
investigated, and in various taxa (e.g.~Kagata et al.~2012, Zhang et
al.~2014, Südekum et al.~2016), the influence of food limitation on this
elemental mass-balance has been poorly documented.

Yet, there are strong grounds to hypothesize a link between food
limitation and elemental mass balance at the individual level. Digestion
and assimilation of food are required for both maintenance, growth and
can vary in efficiency and cost energy. From an evolutionary standpoint,
frequent food limitation cycles might favor individuals which have
higher assimilation and/or growth efficiency to compensate for food
limitation (Pfeiffer et al.~2001, Roller \& Schmidt 2015). When energy
and/or biomass intake rates are below maintenance level, the individual
has an advantage to increase assimilation and/or growth efficiency or to
reduce maintenance and growth requirements (Glazier 2002, Hou et
al.~2015). However, whether such an improvement in assimilation
efficiency is possible depends on the energetic costs associated with
this improvement if there is any, and if the energy intake covers it
while being low. If energy intake is sufficient, the individual could
invest more energy in digestion, decreasing the amount of wastes
produced, something that could be observed in several species (Ali et
al.~2021, Windell 1978, Clauss 2013). But if energy intake is too low,
the digestion function could be impaired, leading to low assimilation
efficiency. Thus, food limitation might trigger higher assimilation
efficiency, up to the point where the cost of assimilation can not be
met anymore because of low intake. There were contrasted results on this
topic

This reasoning can be broadened to specific elements which, like energy,
are necessary for growth and maintenance. For example, nitrogen (N) and
sulfur (S) compose protein, phosphorus (P) is central to metabolic
reaction and nucleic acids, potassium (K), sodium (Na), calcium (Ca) and
magnesium (Mg) are important for osmoregulation and signaling. Food
limitation might change the elemental composition of the animal and of
the wastes that it produces (ref). Indeed, there could be
element-specific increase in assimilation efficiency in food limitation
conditions which might however depend on the content of this element in
the resource. A resource rich in N, although provided in low amounts to
an individual, might be sufficient to support its requirement in N,
leading to no investment in additional N-digestion efficiency. The
contrary is true for resource low in N, and the same reasoning might
apply to the other elements. In turn, these dynamic elemental
modifications of the animal's body and waste composition might alter the
conditions in which its resource is living, in a way that might change
both its quality and productivity.

These individual level processes can also scale up to population level
processes. The functional response links resource density to per capita
intake rate. If there is an additional link between per capita intake
rate and assimilation efficiency, then the resource density and the
assimilation efficiency could be constrained to a certain relationship.
In environments where resource levels vary, for example seasonally, this
link between resource density and waste nutrient fluxes could affect the
temporal dynamics of nutrient cycling. Insects and arthropods represent
the major parts of terrestrial animal biomass (Bar-on et al.~2018). The
boom bust cycles that insects often show could be the conditions where
these processes would occur.

Here we investigate the link between food limitation and elemental mass
balance experimentally in a polyphagous insect larvae (Spodoptera
littoralis) fed an artificial diet at discrete and restrictive levels.
We investigated total mass balance, as well as P, Na, Ca, Mg, K and S
mass balance.

\section{Materials and methods}\label{materials-and-methods}

In order to investigate the link between food limitation and the
elemental mass balance, we tested submitted 7th instar Spodoptera
littoralis larvae to contrasted levels of food limitations, and measured
intake, growth, and egestion rates as well as the chemical composition
of the food, the larvae and the frass.

\subsection{Study system}\label{study-system}

We chose to work with the cotton leafworm Spodoptera littoralis, a
polyphagous lepidopteran that is easily reared on artificial food, which
makes it a suitable model to study the effect of food limitation. The
frass represents the total digestive and excretory wastes which make the
nutritional balance easier. Moreover, the exuviates being consumed,
there was no need to account for the exuviates chemical composition.
This species has six to seven larval instars depending on environmental
conditions (Baker \& Miller, 1974). Individuals from a laboratory strain
were reared on a semi-artificial diet at 23 °C, 60--70 \% relative
humidity, and a 16:8 light/dark cycle. Individuals were isolated at
their 7th and last instar moult before pupation and subsequently
submitted to various levels of food. This stage has the advantage of
producing enough frass to enable individual frass chemical analysis.\\
We choose to distribute a homogen artificial diet that the insect can
not sort. The diet was a hydrogel with composition detailed in table 1.

\subsection{Experiments}\label{experiments}

6 th instar larvae were isolated in individual 15mL circular
polyethylene boxes and provided with ad libitum food until molt
completion. The 7th instar larvae were reared individually in the same
boxes. Individuals were weighed at the beginning of their 7th instar and
three days later, at the end of the experiment.

Each of the 400 individuals was randomly assigned to one of the 5 food
provision levels (120, 240, 360, 480 and 900 mg of fresh weight). From
their 7th instar molt, individuals were given this fixed amount of food
every day, for three days, except when prepuping occurred. Leftovers and
frass were collected everyday, stored at -20°C until pooling with other
samples from the same individual, dried for 72 hours at 60°C in an oven,
and their dry mass was measured. At the end of the experiment, 50\% of
individuals from each treatment were sacrificed at -20°C, dried for 72
hours at 60°C in an oven, and their dry mass measured. The other half of
individuals were used to investigate the effect of food treatment on
mortality, emergence proportion and mass at emergence. For logistics
reasons, to complete the experiment for the 400 individuals, 40
individuals were taken every week for 10 weeks with 8 individuals for
each of the 5 food provision levels.

\subsection{Chemical analysis}\label{chemical-analysis}

Food subsamples were collected for chemical analysis. Food leftovers,
caterpillars, and individual frass samples were dried for 72 hours at
60°C in an oven, and ground to a fine powder using a ball grinder. C and
N were measured using an elemental analyzer.

Samples were microwave-digested (Milestone 1200 Mega,Milestone Inc.,
USA) in Teflon bombs using a mixture of HNO3 and HCl and subsequently
measured for P, K, Mg ,Ca , S and Na using ICP-OES.

\subsection{Growth and mass balance
metrics}\label{growth-and-mass-balance-metrics}

From the weight and elemental content of food and frass, and individual
body masses, we were able to compute the growth and net and
mass-specific ingestion and egestion rates (total and by element), the
digestibilities of food and of elements (\% of food or element that were
assimilated), and the growth efficiency (\% of food that resulted in
growth). Computation and units are given in table 2.

\section{Results}\label{results}

\section{Discussion}\label{discussion}

\section{References}\label{references}

\end{document}
